\documentclass{article}

\usepackage{graphicx} % Required for inserting images
\usepackage{authblk} % For affiliations
\usepackage{enumerate} % For ordered lists
\usepackage{parskip} % Adds white space between paragraphs
\usepackage{orcidlink} % Insert ORCID links
\usepackage{abstract} % Customise abstract
\usepackage{xurl} % To add URLs (which will line break)
\usepackage{hyperref} % To add URLs
\usepackage{geometry} % Tables
\usepackage{booktabs} % Tables (styling)
\usepackage{tabulary,lipsum} % Tables (e.g. J/C/L/R column types)
\usepackage{float} % Tables (prevent repositioning)
\usepackage{glossaries} % Glossary
\usepackage{enumitem} % Use alternatives to 123 in ordered list
\usepackage{forest} % To create directory tree
\usepackage{xcolor} % For highlights and shading
\usepackage{soul} % For highlights
\usepackage{framed} % To make shaded abstract
\usepackage{enumitem,amssymb} % To make check box lists
\usepackage{multicol} % Columns
\usepackage{titlesec} % To make alternative section titles
\usepackage{clock} % Create clock symbol
\usepackage[clock]{ifsym} % Create clock symbol

% Checkboxes
\newlist{todolist}{itemize}{2}
\setlist[todolist]{label=$\square$}

% Change font
\usepackage{cmbright}

% Activate
\makeglossaries

% References using biblatex
\usepackage[
    sorting=none,
    autocite=superscript
]{biblatex}
\bibliography{references}

% Set depth of table of contents
\setcounter{tocdepth}{2}

% Blue highlight
\definecolor{highlightblue}{HTML}{e8f2f6}
\DeclareRobustCommand{\hlblue}[1]{{\sethlcolor{highlightblue}\hl{#1}}}

% Sections of text indicating whether need to record time taken or not
\newcommand{\timeyes}[1][]{
    \vspace{-3mm} \hspace{11mm}
    \hlblue{Record in logbook with time taken #1} {\small \showclock{3}{00}}}
\newcommand{\timeno}[1][]{
    \vspace{-3mm} \hspace{11mm}
    \hlblue{Record in logbook without timing #1}}

% Create shaded abstract
% ---------------------------------------------------
% Credit to JamesT (https://tex.stackexchange.com/questions/675419/text-is-not-centered-with-colorbox)
\definecolor{abstractcolour}{HTML}{D1E6EE}
\colorlet{shadecolor}{abstractcolour}
\renewenvironment{abstract} 
     {
    \vfill%
      \list{}{
        \setlength{\leftmargin}{.3cm}%
        \setlength{\rightmargin}{\leftmargin}%
      \vfill%
      }%
      \item\relax}
 {\endlist}
\setlength{\parindent}{0mm} 
% ---------------------------------------------------

% Create shaded headings
% ---------------------------------------------------
% Logo is #1982AB. Chose paler colous for the heading shading.
\definecolor{section}{HTML}{A3CDDD}
\newcommand{\colorsection}[1]{\colorbox{section}{\parbox{\dimexpr\textwidth-11\fboxsep}{#1}}}
\titleformat{name=\section}[block]
    {\normalfont\Large\bfseries}
    {\thesection}
    {1em}
    {\colorsection}

\definecolor{subsection}{HTML}{A3CDDD}
\newcommand{\colorsubsection}[1]{\colorbox{subsection}{\parbox{\dimexpr\textwidth-13\fboxsep}{#1}}}
\titleformat{name=\subsection}[block]
    {\normalfont\large\bfseries}
    {\thesubsection}
    {1em}
    {\colorsubsection}

\definecolor{subsubsection}{HTML}{D1E6EE}
\newcommand{\colorsubsubsection}[1]{\colorbox{subsubsection}{\parbox{\dimexpr\textwidth-14\fboxsep}{#1}}}
\titleformat{name=\subsubsection}[block]
    {\normalfont\normalsize\bfseries}
    {\thesubsubsection}
    {1em}
    {\colorsubsubsection}
% ---------------------------------------------------

\begin{document}

% Title page and TOC
\title{
    % Move higher on page
    \vspace{-1cm}
    % Include image then 1cm gap
    \includegraphics[width=9cm]{images/stars_logo_blue_text.png}\\[1cm]
    % Title
    \textbf{Protocol for assessing the computational reproducibility of simulation models on STARS}
}

% Authors
\author[1]{\orcidlink{0000-0002-6596-3479} Amy Heather}
\author[1]{\orcidlink{0000-0003-2631-4481} Thomas Monks}
\author[2]{\orcidlink{0000-0001-5274-5037} Alison Harper}
% Include line break (to prevent name breaking over two lines)
\author[2]{\\ \orcidlink{0000-0002-2204-8924} Navonil Mustafee}
\author[3]{\orcidlink{0000-0003-1263-2286} Andy Mayne}

% Affiliations (ensure update with numbers used above)
\affil[1]{\footnotesize University of Exeter Medical School, Exeter, UK}
\affil[2]{\footnotesize University of Exeter Business School, Exeter, UK}
\affil[3]{\footnotesize Taunton and Somerset NHS Foundation Trust, UK}

% No date
\date{}

\maketitle

\textbf{Publication date:} \hl{Add prior to submission}

% Add space before abstract
\vspace{0.5cm}

% Create abstract in a shaded box
\begin{shaded}
    \begin{abstract}
        This protocol outlines how we plan to reuse available artefacts to reproduce results from published simulation studies. This forms part of the project STARS: "Sharing Tools and Artefacts for Reproducible Simulations in healthcare". It will be utilised to assessed the computational reproducibility of published discrete-event (DES) simulation models in Python and R.
    \end{abstract}
\end{shaded}

% Add space after abstract
\vspace{0.5cm}

This protocol is archived as a pre-registration. Haroz 2022 identifies the Open Science Framework (OSF, \url{https://osf.io/}) and Zenodo (\url{https://zenodo.org/}) as suitable platforms for pre-registration.\autocite{haroz_comparison_2022} In this case, Zenodo will be used as this is where other materials already exist for the STARS project, and so it can be stored alongside them in a Zenodo "community".

\newpage
\tableofcontents

% Glossary
% \section{Glossary}

% These only display if you use like \gls{reuse} 

\newglossaryentry{computational reproducibility}{name={computational reproducibility}, description={Definition of computational reproducibility... may also call  or "reproducibility test" or }}

"replicated computational results" acm

"push button replication" https://journals.plos.org/plosone/article?id=10.1371/journal.pone.0209416

"narrow replication" (https://link.springer.com/article/10.1007/s40273-019-00836-y)

\newglossaryentry{reuse}{name={reuse}, description={Definition of reuse}}

\newglossaryentry{reproduce}{name={reproduce}, description={Definition of reproduce}}

\newglossaryentry{replicate}{name={replicate}, description={Definition of replicate}}

% Display glossary (without adding a title)
\renewcommand{\glossarysection}[2][]{}
\printglossaries

example definitions:

https://www.niso.org/standards-committees/reproducibility-badging - slide 5

% Main body of protocol
\section{Introduction to the STARS project}

This protocol is part of the project STARS: ”Sharing Tools and Artefacts for Reusable Simulations in healthcare”. The aim of STARS is to...

\textbf{TO DO:} Write this section. Mention the aim, and refer to the key papers/pilot work, and then how that relates to this protocol (to set the scene). Keep it brief.

\textbf{Status as of submission:} X (name) selected Shoaib and Ramamohan 2021\autocite{shoaib_simulation_2022} as one of the six studies, and emailed to ask for open license, which they have added to repository.

towards stars paper\autocite{monks_towards_2024}

\section{Selection of simulation models}

This protocol will be used within two work packages on STARS:
\begin{itemize}
    \item Work package 1 - to reproduce six published discrete event simulation models by external authors
    \item Work package 3 - to reproduce two simulation models produced by/in collaboration with members of the STARS team
\end{itemize}

\textbf{TO DO:} Remove mention of work packages, and just focus on the work being done (as work packages just relevant to us, not external) as that would make this clearer

For work package 1, the six models will be chosen from those identified by Monks and Harper 2023\autocite{monks_computer_2023} in their paper "Computer model and code sharing practices in healthcare discrete-event simulation: a systematic scoping review". Their review of discrete-event simulation models in healthcare identified 47 studies citing an openly available model that was used to generate their results. When selecting six of these models, we have two criteria: (i) The study has an open license (either already published, or added upon request from the STARS team), and (ii) That a balance of Python and R models are chosen.

For work package 3, the models will be developed within or with support from the STARS team, with a team member acting as a holdout to conduct the reproducibility test.

For WP1, email the authors. If email not active (e.g. rebounds) then search online to try and find most recent email. Email, if no response after 2 weeks then contact again.

Write this properly, base on and cite:
\begin{itemize}
    \item "PBR researchers email the corresponding author and at least one additional author (if applicable) to inform them that they will be conducting a PBR of their paper, include the PBR protocol,"\autocite{berkeley_initiative_for_transparency_in_the_social_sciences_guide_2022}
\end{itemize}

\section{Quarto website, logbook and timing}

A Quarto website will be produced using our template (\url{https://github.com/pythonhealthdatascience/stars_reproduction_template}) to compile information on the reproduction of the article. This includes the notebooks (.ipynb or .Rmd) producing the items in the scope, as well as a chronological log of work using Quarto blog posts.

Within each post in the log book it should include the researcher name and date along with:
\begin{itemize}
    \item Comprehensive record of tasks, along with time spent (if applicable)
    \item Issues, barriers and enablers to reproduction
    \item Solutions to problems
    \item Timing and progress in reproducing each item in the scope (so we know what is completed when)
    \item Relevant links or links to files at that point (e.g. Git commit hash), with relevant files like the script and output files.
    \item Explanation of why things were done
    \item Notes on critical and non-critical issues to be addressed (such as in a to do list)
    \item If it makes more sense to include detailed descriptions elsewhere (such as in a script or notebook itself), then just be sure to link to that version of that file (such as via the Git commit history)
\end{itemize}

As suggested by Ayllón et al. 2021\autocite{ayllon_keeping_2021} in their guidelines for keeping modelling notebooks, these posts will be daily, dated, chronological entires. Tags will be used to help indicate the activity on each day, and enable posts to be filtered by activity (although tags are free to be chosen by the researcher not related to a particular framework). Keeping a detailed log will support later understanding of what was done, and support preparting of final documents like the reproduction report.

For tasks relating to the reproduction of the article, a record of time taken should also be included within the log alongside each activity (e.g. 12:10 to 12:45). Throughout the sections below, a note is made as to whether the task should or should not be timed, but to summarise:
\begin{itemize}
    \item Assessment of computational reproducibility is mostly timed (with a few exceptions), as noted below alongside each task. These times will be monitored, with a maximum of 40 hours allowed for attempting to reproduce the study, as in Krafczyk et al. 2021.\autocite{krafczyk_learning_2021} This cut-off is implemented as we anticipate there would be little more to learn from spending longer than that time on reproducing a single study.
    \item Creation of the reproduction package is timed, as time is often cited as a barrier to making research open, and so it is helpful to understand time taken on this task, but it is not capped with a time limit.
    \item The later stages are not timed. These are evaluation of the study against reproduction badges, code guidelines and reporting guidelines, and writing up the reproduction test report.
\end{itemize}
\newpage
\section{Assessment of computational reproducibility} \label{sec:reproduce}

\subsection{Set-up}

\subsubsection{Inform the authors}
\timeno

If the authors have not already been informed (such as via an email to request an open license is added to the repository), then the researcher should let the corresponding author know about this study. This email should inform the author that we will be conducting an assessment of the computational reproducibility of their study, and can include a copy of the protocol. If the email appears inactive due to rebounds, the researcher should search online to try and identify a recent email address for any of the study authors.

\subsubsection{Create repository using template}
\timeno

Use the STARS computational reproducibility template to create a new repository.

\begin{itemize}
    \item Template: \url{https://github.com/pythonhealthdatascience/stars_reproduction_template}. To use it, select the "Use this template" button.
    \item Organisation: \url{https://github.com/pythonhealthdatascience}
    \item Repository name: stars-reproduce-firstauthorsurname-publicationyear
    \item Description: Assessing the computational reproducibility of [firstauthor et al. publicationyear] as part of STARS.
    \item Public repository
\end{itemize}

\subsubsection{Upload journal article and all artefacts to the repository}
\timeyes

Upload all available materials from the article to the repository. This could include:
\begin{itemize}
    \item Journal article
    \item Supplementary materials
    \item Code (for example, copying the content of a GitHub repository)
    \item Protocol
\end{itemize}

\subsubsection{Select license}
\timeyes

Select a license for the repository. At this stage, this should be as permissive as possible, whilst being compatible with the license used by the study authors. For instructions on how to add a license to a repository, see \url{https://docs.github.com/en/communities/setting-up-your-project-for-healthy-contributions/adding-a-license-to-a-repository}.

\subsection{Scope of reproduction}

\subsubsection{Read the journal article}
\timeyes

Read through the journal article and supplementary materials (but not yet looking into the code or data). Consolidate this by adding a short description of the article to the Quarto book. This should be relatively concise, and cover the key methods and results. It might, for example, include any model flow diagram provided in the paper, any model parameters mentioned, and the model type and software. This can also be updated later.

\subsubsection{Make consensus decision on scope}
\timeyes[for scoping by primary researcher (but not including time of other researchers reading the article, or time spent discussing and coming to a consensus on the scope)]

The next step is to define the scope of the reproducibility study - in other words, what parts of the paper you intend to reproduce. The scope should include:
\begin{enumerate}[label=(\alph*)]
    \item All tables and figures within the manuscript that contain results from the simulation, and-
    \item Any other key results in the paper, that are not otherwise captured within the tables and figures. "Key results" can be considered things that are mentioned within the abstract or conclusion, or feature prominently within the results section.
\end{enumerate}

This does not need to include tables and figures in the Appendices (unless these are identified as "key results"). This method of defining the scope is based on that used by Wood et al. 2018.\autocite{wood_replication_2018, wood_push_2018}

Whilst the rest of this protocol is conducted by a single member of the STARS team, the scope of the reproducibility study will be defined as a consensus decision, with at least two team members reading the article and suggesting the scope independently, and then agreeing together on what is appropriate.

\subsubsection{Compile items in scope}
\timeyes

Once the scope has been decided, upload each item to the repository.

\begin{itemize}
    \item For tables, download a CSV version of each if available. Otherwise, convert the tables into CSV format.
    \item For figures, download the highest-quality version of each figure that is available. Then extract data from the figures. \hl{Do we want to do this? Do we trust the software?} Software available: \url{https://plotdigitizer.com/best-plot-digitizer}. Good options are:
    \begin{itemize}
        \item WebPlotDigitizer (\url{https://automeris.io/posts/}) - in active development, version 4 and below are GNU AGPL v3, version 5 (published 14 May 2024) and onwards will be closed source.
        \item Enguage Digitizer (\url{https://markummitchell.github.io/engauge-digitizer/}) - no further development since 3 years ago, GNU GPL v2
    \end{itemize}
    \item For results described in the text (but not captured in a table or figure), record in a format appropriate to then later compare against (for example, within a CSV).
\end{itemize}

\subsubsection{Archive scope and supporting documents on Zenodo}
\timeyes

Describe the scope within the Quarto book, providing justification for any "key results" chosen, and including reference to the uploaded figures and data for the items. Any tables and figures not being produced (for example, as they don't contain results from the simulation) should also be noted.

With the organisation linked to Zenodo, toggle Zenodo to preserve that repository, and then create a release on GitHub, which Zenodo will then automatically download and register with a DOI. This release should serve as a public registration of the intended scope of the reproducibility study, and archiving the repository at this point (prior to having started using or really looking at the the code).

As stated in the "Guide for Accelerating Computational Reproducibility (ACRe) in the Social Sciences", it is important that the scope is defined at the start of the study, and publicly archived so as not to be amended during the course of the study.\cite{berkeley_initiative_for_transparency_in_the_social_sciences_guide_2022}

\subsection{Familiarise with artefacts}

\subsubsection{Describe code/data}
\timeyes

Browse through any code and any data that you uploaded to the repository. In the Quarto book, add a tree of the uploaded files, along with one-sentence description of each file.

This is as suggested by Ayllón et al. 2021\autocite{ayllon_keeping_2021} in their guidelines for keeping modelling notebooks

\subsubsection{Search for code that produces items in scope}
\timeyes

As in Krafczyk et al. 2021,\cite{krafczyk_learning_2021} look over the code and any data, and attempt to identify sections that correspond with items in the scope. Record this within the log.

\subsection{Set up environment}

\subsubsection{Identify dependencies}
\timeyes

To ensure a reproducible research environment, we need to know:
\begin{itemize}
    \item The operating system used (e.g. Windows, Mac, Linux) and its version 
    \item The software used (e.g. Python, R)
    \item The software packages used and their versions\autocite{the_turing_way_community_turing_2022}
\end{itemize}

However, it is likely that papers may not include all the information required (such as versions used). In which case:
\begin{itemize}
    \item No operating system - use Linux
    \item No operating system version - use closest to publication date \hl{Do we want to be this stringent?}
    \item No package list - look at the libraries/packages imported within the scripts
    \item No package versions - use closest to publication date
\end{itemize}

\subsubsection{Create environment}
\timeyes

\hl{Evaluate these options. Consider whether they control OS. Consider whether they control date. But also - whether we want that. How casual vs. standardised on this we want to be.}

One suggestion is to install packages and their dependencies at a point in time no later than publication date. Not all solutions below guarantee this though. The publication date is used as could have re-run code during approval process, but know not after publication.
\begin{itemize}
    \item Anecdotally, this could be particularly handy for R which often tries to get the most recent packages
\end{itemize}

Another suggestion is to match operating system to developer. Again, not all solutions below guarantee this. This will also be relevant later if want to test reproduction package on different operating systems.

Python -
\begin{itemize}
    \item Conda, VirtualEnv, etc.
    \item Docker - to do development inside container, VSCode Dev Containers extension allows you to open folder inside container - \href{https://code.visualstudio.com/docs/devcontainers/containers}{source 1}
\end{itemize}

R -
\begin{itemize}
    \item Renv
    \item Posit Public Package Manager - can use Snapshot (earliest is Oct 2017, and 5 most recent versions of R), for Linux can install binary packages (which is much quicker, as usually R installs from source rather than binary unlike for Windows and Mac which makes it really slow) - \href{https://packagemanager.posit.co/client/#/repos/cran/setup}{source 1}, \href{https://docs.posit.co/faq/p3m-faq/#frequently-asked-questions}{source 2}
    \item Groundhog - can go back to R 3.2 and April 2015 (and apparently can patch to go earlier) - \href{https://www.brodrigues.co/blog/2023-01-12-repro_r/}{source 1}
    \item miniCRAN - \href{https://learn.microsoft.com/en-us/sql/machine-learning/package-management/create-a-local-package-repository-using-minicran?view=sql-server-ver16}{source 1}
    \item Docker - requires license for non-academic (e.g. NHS) use - but Podman can drop in as replacement. To do development inside a container isn't natively supported by RStudio but can use RStudioServer via Rocker. By default, it runs in ephemeral mode - any code created or saved is lost when close - but you can use volume argument to mount local folders - \href{https://towardsdatascience.com/running-rstudio-inside-a-container-e9db5e809ff8}{source 1}
\end{itemize}

\subsubsection{Update license if necessary}
\timeyes

As above, we want to use a license that is as permissive as possible whilst being compatible with the license used by the author, as well as now the \textbf{licenses of any packages} used within the project. To identify the licenses used:
\begin{itemize}
    \item In Python, you could use the package "pip-licenses" (\url{https://pypi.org/project/pip-licenses/})
    \item In R, you can run the command \texttt{installed.packages(fields = "License")}
\end{itemize}

If it is not possible to use a completely permissive license like MIT, explain why (either due to author license or package licenses).

\subsection{Attempt to reproduce items in scope}

\subsubsection{Use provided code to produce notebook creating items in scope}
\timeyes[(excluding computation time at the researcher's discretion - for example, excluding time where a simulation is set to run for 5 minutes whilst the researcher makes a cup of tea)]

Run the code and attempt to produce the items in the scope. Insert the code into a \textbf{notebook} (.ipynb or .Rmd) as this enables you to easily share the code and produced outputs from the scope. Label each output clearly, as it relates to the manuscript. Whilst running this code, the researcher should be troubleshooting issues and evaluating reproduction success, as below.

When using their code, this should be kept separate from the unchanged copies of code that were downloaded at the start of the study. For example, copying over their script into a new notebook in a different folder (with separate folders for artefacts from the article, and for results produced by the researcher reproducing the article).

\subsubsection{Troubleshoot issues}
\timeyes[(including searching online, looking over artefacts, emailing the author, etc.)]

If there are difficulties in running the code or producing the desired outcomes, researchers should attempt to troubleshoot through changes to the environment or scripts. Examples of changes include:
\begin{itemize}
    \item Correcting paths to files
    \item Correcting the versions of software, or adding missing packages or libraries
    \item Fixing errors in the code
    \item Adding code to produce an item in the scope, if not otherwise provided
\end{itemize}

In allowing modification and writing of code, our intention is that researchers try \textbf{as much as possible} to attempt to reproduce from the scope. The allowance of writing new code is similar to the approach of Krafczyk et al. 2021\autocite{krafczyk_learning_2021} and the ACRe project\autocite{berkeley_initiative_for_transparency_in_the_social_sciences_guide_2022}.

If however, they remain unable to run the code, or have large discrepancies for any items, or any completely unable to reproduce any items, they should then \textbf{contact the original author}. This email should:
\begin{itemize}
    \item Recap of project (as will have emailed them before when started)
    \item Link to preliminary report with documented attempt and list of issues that require resolution, Make sure description of problem is specific (e.g. identifying line in paper and place in code where think something is missing, or where an issue is occuring)
    \item Ask for suggestions on alternative course of action for issues, or for the complete code/data if missing.
\end{itemize}

If there is no response in two weeks, the researcher should contact them again. If there is still no response two weeks later, this can be marked as no response. When emailing authors, it is suggested to follow the guidance on language and adapt from the \textbf{email template}s provided by the ACRe in the chapter "Guidance for Constructive Communication Between Reproducers and Original Authors" of their guide.\autocite{berkeley_initiative_for_transparency_in_the_social_sciences_guide_2022}

The allowance of contacting authors is similar to the approaches of several studies,\autocite{krafczyk_learning_2021,wood_push_2018,berkeley_initiative_for_transparency_in_the_social_sciences_guide_2022,hardwicke_analytic_2021,konkol_computational_2019} with a maximum of four weeks for responses as in Konkol et al. 2018\autocite{konkol_computational_2019}. This approach does however differ from Laurinavichyute et al. 2022\autocite{laurinavichyute_share_2022} who did not contact authors, since they considered reproducibility to be only about the available data and procedures and not anything shared privately.\autocite{laurinavichyute_share_2022}

\subsubsection{Assess reproduction success}
\timeyes

For each item in the scope, the researcher should decide whether it has been successfully reproduced. This decision can be supported by calculating the percentage difference in results between the manuscript and the researcher. As reported by Wood et al. 2017,\autocite{wood_push_2018, wood_replication_2018} a meaningful difference in a value will vary between studies, and so it is difficult to set a single rule on what is or is not a minor difference. As such, researchers should follow a similar approach to Schwander et al. 2021,\autocite{schwander_replication_2021} considering whether the figure is reproducible at varying levels of percentage error (5\%, 10\% and 20\%). However, they should then use their judgement to decide whether the item has been reproduced. This is similar to one of the definitions proposed by McManus et al. 2019\autocite{mcmanus_can_2019} - "\textit{Results... vary only by XX\% compared to the original, AND are consistent with the original conclusions}" - incorporating both numerical comparison and allowance for variability in whether this constitutes a meaningful difference from the original results.

The researcher should not use a definition of "partial success". Instead, they should decide whether or not each item (e.g. table, figure, or section of text) was successfully reproduced or not as a whole (although can include a description to provide more nuance to that decision). As an example, if it is possible to produce a table with some numbers being a match or very similar, but some numbers being substantially different, then this would be classed as having \textbf{not} been successfully reproduced. If however all aspects of the item were reproduced with reasonable similarity, this can be classed as \textbf{successful} reproduction.

As in Henderson et al. 2024,\autocite{henderson_reproducibility_2024} figures are compared based on their numeric content. This can be based on numbers extracted from the figure (as below), or based on visually comparing the figures (in which case, percentage difference cannot be used to support any judgement of reproduction success). Researchers should be unconcerned by minor differences in presentation (such as font), with regards to evaluating reproduction success.

The overall reproduction success of the paper can be described as the proportion of items from the scope that were determined to be reproduced (e.g. 4 out of 5, 80\%). This provides more context than just stating that a paper was or was not reproduced. When describing the reproduction success, it is important to include context on troubleshooting steps required (such as writing new code or contacting the author)

In assessing reproduction success, it is important to note (as in Laurinavichyute et al. 2022\autocite{laurinavichyute_share_2022} and Wood et al. 2018\autocite{wood_push_2018}) that the focus is not on the quality or robustness of the original results, or whether the main claims of the study are consistent. Instead, the focus is on whether it was possible to reproduce the article's results within a reasonable margin of error (given that we do expect a little variation, since discrete-event simulations are stochastic models, and may not have been fully controlled using random seeds or with any environment differences).

\subsubsection{Archive on Zenodo}
\timeyes

Once all of the above is completed, you should archive the repository on Zenodo through creation of a new GitHub release. The repository should already be set up to sync with Zenodo from the earlier stage when we archived the scope, so this release will create an updated version of the archived repository, which will sit alongside the prior version.
\section{Second stage: evaluation}

This section is completed \textbf{after} the attempted reproduction (so as to not interfere with timings).

\textbf{Remember!} Record progress in your \textbf{logbook} and \textbf{time spent} on each task.

\textbf{Important:} This evaluation is based on the \textbf{original} journal article or repository from the author (as in 'original\_study/'), and not on the repository that was created whilst reproducing this study ('reproduction/'). If the original study had multiple repositories to choose from (e.g. development and archived code, both prior to publication date), remember to \textbf{refer to both of them} if there are any differences between them.

\textbf{Getting a second opinion:} If the researcher is uncertain about any criteria, they should note these in the \textbf{logbook}. Any criteria that were \textbf{unmet} or \textbf{uncertain} should then be discussed with at least one other researcher on the project to get a second opinion. Record the discussion (and its timing) in the logbook, and explain and justify the choices for uncertain items.

\vspace{0.5cm}
\subsection{Badges} \label{sec:badges}

Several organisations and journals have developed badges which can be displayed alongside a research article to indicate how open and potentially reproducible it is, as detailed in Appendix \ref{appendix:badges}. These include the National Information Standards Organisation (NISO),\autocite{association_for_computing_machinery_acm_artifact_2020} the Association for Computing Machinery (ACM),\autocite{association_for_computing_machinery_acm_artifact_2020} the Institute of Electrical and Electronics Engineers (IEEE),\autocite{institute_of_electrical_and_electronics_engineers_ieee_about_nodate} the Center for Open Science (COS)\autocite{blohowiak_badges_2023}, and the journal Psychological Science.\autocite{hardwicke_transparency_2023,association_for_psychological_science_aps_psychological_2023}

We will evaluate the original study artefacts (repository) against badges that relate to code (and not those specific to data), due to the nature of DES models (where ``data" is often just parameters as part of the model script, with perhaps a few additional parameters in a separate data file within the repository). The \textbf{badges we will evaluate against} are:
\begin{itemize}
    \item ``Open objects" badges:
    \begin{itemize}
        \item NISO ``Open Research Objects (ORO)" and ``Open Research Objects - All (ORO-A)"\autocite{niso_reproducibility_badging_and_definitions_working_group_reproducibility_2021}
        \item ACM ``Artifacts Available"\autocite{association_for_computing_machinery_acm_artifact_2020}
        \item COS ``Open Code"\autocite{blohowiak_badges_2023}
        \item IEEE ``Code Available"\autocite{institute_of_electrical_and_electronics_engineers_ieee_about_nodate}
    \end{itemize}
    \item ``Object review" badges:
    \begin{itemize}
        \item ACM ``Artifacts Evaluated - Functional" and ``Artifacts Evaluated - Reusable"\autocite{association_for_computing_machinery_acm_artifact_2020}
        \item IEEE ``Code Reviewed"\autocite{institute_of_electrical_and_electronics_engineers_ieee_about_nodate}
    \end{itemize}
    \item ``Reproduced" badges:
    \begin{itemize}
        \item NISO ``Results Reproduced (ROR-R)"\autocite{niso_reproducibility_badging_and_definitions_working_group_reproducibility_2021}
        \item ACM ``Results Reproduced"\autocite{association_for_computing_machinery_acm_artifact_2020}
        \item IEEE ``Code Reproducible"\autocite{institute_of_electrical_and_electronics_engineers_ieee_about_nodate}
        \item Psychological Science ``Computational Reproducibility"\autocite{hardwicke_transparency_2023,association_for_psychological_science_aps_psychological_2023}
    \end{itemize}
\end{itemize}

The researcher should use the \textbf{provided template} (evaluation/badges.qmd) to assess whether the artefacts from the original study meet the criteria for each of these badges. A \textbf{binary} decision is made for each criteria (as being either met or not met).

\vspace{0.5cm}
\subsection{STARS framework}

The artefacts (repository) associated with the original study will be evaluated against the \textbf{STARS framework}, which has essential and optional recommendations for sharing research artefacts from healthcare simulation studies. This framework was designed by Monks et al. (2024)\autocite{monks_towards_2024} to complement and build on general open science recommendations from the Turing Way,\autocite{the_turing_way_community_turing_2022} Taylor et al. (2017),\autocite{taylor_open_2017} and the Open Modelling Foundation (OMF) minimal and ideal reusability standards.\autocite{the_open_modeling_foundation_omf_reusability_2024}

The researcher should use the \textbf{provided template} ('evaluation/artefacts.qmd') to assess whether the artefacts from the original study meet the recommendations from this framework. Each criteria are evaluated as being ``fully", ``partially" or ``not met".

\vspace{0.5cm}
\subsection{Reporting guidelines} \label{sec:reporting}

The \textbf{journal article} will be evaluated against two reporting guidelines for discrete-event simulation studies:
\begin{itemize}
    \item STRESS-DES: Strengthening The Reporting of Empirical Simulation Studies (Discrete-Event Simulation) - Monks et al. (2019)\autocite{monks_strengthening_2019}
    \item The generic reporting checklist for healthcare-related discrete event simulation studies derived from the the International Society for Pharmacoeconomics and Outcomes Research Society for Medical Decision Making (ISPOR-SDM) Modeling Good Research Practices Task Force reports - Zhang et al. (2020)\autocite{zhang_reporting_2020}
\end{itemize}

The researcher should use the \textbf{provided template} ('evaluation/reporting.qmd') to assess whether the criteria from these guidelines are met by the journal article (including the supplementary material, although not including the code unless the article specifically refers to it for providing particular information). Each criteria are evaluated as being ``fully", ``partially" or ``not met", with detailed evidence provided to support these claims (such as quotations from the article). If a criteria is not met by the original study, the researcher is welcome to make a suggestion in the evidence column of what they think the likely answer for that criteria might be.
\section{Reproduction test reports}

\hlblue{\textbf{Grant:} Write RCR reports. \textbf{To discuss though!:} Got lots and lots of examples of write ups for this sort of thing, and identified opportunities for publishing individually (ReScience)}

\textbf{TO DO:} Write this section. Need to reflect on whether it can consistute an RCR review or not. And what we want to do with it, or if just archived, or if plan to write up in some way or another.

Tom: Finally, we will map data from our best practice review, reporting guidelines adherence, and our tracked test data to success in badge allocation and the proportion of experiments reproduced.

Produce a plot reporting the percentage of experiments reproduced versus time. - but need to consider this based on how define reproduction success

Final report:
\begin{itemize}
    \item Figure, and describe reproduction team, similar to Krafcyzk(?):\autocite{krafczyk_learning_2021} Krafczyk present time taken to produce each table/figure. Consider approach to this given: * Won't do a yes/no for reproduction, but more nuanced info on it for each item * Not just tables and figures - "key results" - which could be a table/figure from appendices, but could also be specific figures from the text. * A study with more figures and tables will take longer (which is not presented in that figure as its percentage of them done)
    \item Look at examples I've gathered
\end{itemize}

\section{Versioning and Archiving}

This reproduction work will be made openly available throughout the project using GitHub. It will also be archived in Zenodo upon completion.

\textbf{TO DO:} Is there any exception here with the NHS Somerset work, in terms of how will work?

This protocol itself will also be archived as a pre-registration. Haroz 2022 identifies the Open Science Framework (OSF, \url{https://osf.io/}) and Zenodo (\url{https://zenodo.org/}) as suitable platforms for pre-registration.\autocite{haroz_comparison_2022} In this case, Zenodo will be used as this is where other materials already exist for the STARS project, and so it can be stored alongside them in a Zenodo "community".

\textbf{TO DO:} Maybe remove this section, and just mention abut GitHub and Zenodo above in record keeping and reports, and don't need to mention Haroz?

Rough notes from Tom:

Will use GitHub + Zenodo (automated via GitHub actions)
Will containerise with Docker or similar (again, automated via GitHub actions)
Will share via GitHub pages
MIT and CC-BY-4.0 licenses
Online book will describe how to reuse, adapt and reshare our outputs

Create standardised structure for report, and include stuff from the original article within it.

Report should have things like Baykova \autocite{baykova_ensuring_2024} like:
\begin{itemize}
    \item Title of manuscript
    \item Corresponding author
    \item Summary of paper
    \item Available manuscript materials DOI/URL... pre-registratin... manuscript... data and analysis materials
    \item Scope of reproducibiltiy report
    \item Date it was compiled, what it was based on (e.g. draft, pre-print), materials uploaded to (link) as of (date and time of upload)
    \item "I certify that to the best of my knowledge this report is true and accurate"
    \item Who did the report, email, address
\end{itemize}

\section{Testing the protocol}

Test with Tom's paper.

Also maybe IPACS?

Think about what test and why. Is it run through of whole thing, or just particular bits.


% Print bibliography and include in table of contents
\printbibliography[heading=bibintoc]

% Sections after this point are Appendices
\appendix
\section{Appendix: Badges}

\subsection{Current reproducibility badges}

A badge is an that is displayed alongside a published research article. There are currently several different badges from various organisations that can be used to indicate how open and potentially reproducible an article and its artifacts are. These are summarised in Table \ref{table:badges}, with definitions based on those from the "Reproducibility Badging and Definitions" report by the National Information Standards Organisation (NISO).\autocite{niso_reproducibility_badging_and_definitions_working_group_reproducibility_2021}

% "H" prevents repositioning, using float
\begin{table}[H]
\centering
\caption{Summary of reproducibility badging systems, with category definitions adapted from the NISO report.\autocite{niso_reproducibility_badging_and_definitions_working_group_reproducibility_2021}}
% Space between caption and table
\vspace{0.2cm}
% Identifier for the table
\label{table:badges}
% Minimum column width
\tymin=2cm
% Add space between table rows
{\renewcommand{\arraystretch}{1.2}
    % Create table with 3 left-aligned multi-line columns, dynamically sized
    \begin{tabulary}{\linewidth}{@{}LLL@{}}
          \toprule
          Category & Definition & Badges \\
          \midrule
          \textbf{Pre-registration} &
            - &
            \textbullet\ COS "Pregistered"
            \\\addlinespace
          \textbf{Open objects} &
            Digital objects (data, code) permanently archived in public repository with persistent identifier and open license  &
            \textbullet\ NISO "Open Research Objects (ORO)"\newline \textbullet\ ACM "Artifacts Available"\newline \textbullet\ COS "Open Data" and "Open Materials"\newline \textbullet\ IEEE "Code Available" and "Datasets Available"\newline \textbullet\ Springer Nature "Badge for Open Data"
            \\\addlinespace
          \textbf{Object review} &
            Digital objects (data, code) reviewed according to criteria of badge issuer &
            \textbullet\ NISO "Research Objects Reviewed (ROR)"\newline \textbullet\ ACM "Artifacts Evaluated"\newline \textbullet\ IEEE "Code Reviewed" and "Datasets Reviewed"
            \\\addlinespace
          \textbf{Reproduced} &
            Independent party regenerated results using author objects &
            \textbullet\ NISO "Results Reproduced (ROR-R)" \newline \textbullet\ IEEE "Code Reproducible" and "Dataset Reproducible" \newline \textbullet\ Psychological Science "Computational Reproducibility"
            \\\addlinespace
          \textbf{Replicated} & 
            Independent study on same question finds consistent results (potentially with new artifacts and methods) &
            \textbullet\ NISO "Results Replicated (RER)" \newline \textbullet\ IEEE "Code Replicated" and "Dataset Replicated"
          \\\bottomrule
    \end{tabulary}
}
\end{table}

% Table footnotes
\vspace*{-1.5\baselineskip}
\footnotesize Abbreviations: ACM, Association for Computing Machinery; COS, Center for Open Science; IEEE, Institute of Electrical and Electronics Engineers; NISO, National Information Standards Organisation.
\normalsize 
\\

\subsubsection{National Information Standards Organisation (NISO)}

NISO is a US organisation that publishes technical standards for information management. One of their less formal standards if a "recommended practice"  report which suggests best practice - in this case, for "Reproducibility Badging and Definitions". The working group for the report included members from several different institutions including institutions, journals, and archives - includes those who have produced the other badges mentioned in Table \ref{table:badges}. They propose that these standards are universally deployed across scholarly publishing output.\autocite{niso_reproducibility_badging_and_definitions_working_group_reproducibility_2021}

Rough notes:
\begin{itemize}
    \item These badges are not specific to a particular publisher.
    \item Not sure how they are awarded - this example is a conference that established badges using the NISO definitions that they themselves awarded - \url{https://sc21.supercomputing.org/submit/reproducibility-initiative/ad-ae-appendix-process-badges/index.html}
    \item They do not yet have images.
    \item Last project update appears to be June 2022 - \url{https://www.niso.org/standards-committees/reproducibility-badging} - although it is described as an active committee
\end{itemize}

\textbf{TO DO:} Finish write up, and understand whether and how we use these badges, and if it would require us "making" a definition or if NISO is specific enough.

\subsubsection{Association for Computing Machinery (ACM)}

\textbf{TO DO:} Write this section.

https://www.acm.org/publications/policies/artifact-review-and-badging-current

ACM RCR Version 1.1

Replicated Computational Results (RCR)

Not results replicated (as that is independently getting same results without using author-supplied artifacts)

What journals?

Image source: \url{https://www.acm.org/publications/policies/artifact-review-and-badging-current}

\includegraphics[width=3cm]{images/artifacts_evaluated_functional_v1_1.png}
\includegraphics[width=3cm]{images/artifacts_evaluated_reusable_v1_1.png}
\includegraphics[width=3cm]{images/artifacts_available_v1_1.png}
\includegraphics[width=3cm]{images/results_reproduced_v1_1.png}

\subsubsection{Institute of Electrical and Electronics Engineers (IEEE)}

\textbf{TO DO:} Write this section.

IEEE Explore contains content published by IEEE and its partners. Code and datasets can be awarded four types of badge:

\begin{itemize}
    \item Code/Dataset Available
    \item Code/Dataset Reviewed
    \item Code/Dataset Reproducible
    \item Code/Dataset Replicated
\end{itemize}

What journals?

Image source: \url{https://ieeexplore.ieee.org/Xplorehelp/overview-of-ieee-xplore/about-content#reproducibility-badges}

\includegraphics[width=15cm]{images/IEEE_reproducibility_badges.png}

\subsubsection{Center for Open Science (COS)}

\textbf{TO DO:} Write this section.

Image source: \url{https://editorresources.taylorandfrancis.com/the-editors-role/open-research/open-science-badges/}

\includegraphics[width=3cm]{images/Open-Data-OSB.png}
\includegraphics[width=3cm]{images/Open-Materials-OSB.png}
\includegraphics[width=3cm]{images/Preregistered-OSB.png}

\subsubsection{Springer Nature}

\textbf{TO DO:} Write this section.

Image source: \url{https://badgr.com/public/assertions/zfqlozFWT4m5D3TlcDTVNw}

\includegraphics[width=3cm]{images/springer_open_data_badge.png}

\subsubsection{Psychological Science}

In 2013, the Psychological Science journal began using the COS badges for pre-registration, open data and open materials. However, studies that have demonstrated the gaps in these badges - that open data doesn't guarantee reproducibility (with three studies struggling to reproduce articles from the journal), and also that pre-registrations can lack detail and that deviations from the plan may not be disclosed. Following these findings, they have introduced several changes:

\begin{itemize}
    \item \textbf{Changing COS badges into requirements:} Data and materials must be publicly available in a trusted repository
    \item \textbf{Clear documentation:} Required (e.g. README, data dictionary) along with ideally open licenses
    \item \textbf{STAR checks:} Team of Statistics, Transparency, and Rigor (STAR) editors to:
    \begin{enumerate}
        \item Assist editors ad-hoc
        \item Perform transparency checks on all manuscript (lightly before peer-review, in-depth after)
        \item Adjudicate on provision of \textbf{Computational Reproducibility badge}
        \item Conduct random computational reproducibility checks
    \end{enumerate}
    \item \textbf{Collaboration with Institute for Replication (I4R):} I4R \url{https://i4replication.org/} are allowing submission of articles from this journal to them. Verifiers from I4R will write a verification report which is publicly shared, and included in co-authorship on meta-paper that includes the verifications.
\end{itemize}

% Not currently included - would necessitate CC-BY NC license likewise - but mainly just as it's more of a reference to elsewhere so not sure if relevant to include in full
% \include{sections/acre_constructive_communication}

\end{document}
