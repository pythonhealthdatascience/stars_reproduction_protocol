\section{Guidance for Constructive Communication Between Reproducers and Original Authors}
\label{appendix:acre}

\begin{shaded}
    This section is a copy of Chapter 8 from: \textbf{Berkeley Initiative for Transparency in the Social Sciences. Guide for Advancing Computational Reproducibility in the Social Sciences. Sept. 2022. URL: https://bitss.github.io/ACRE/ (visited on 05/15/2024).}\autocite{berkeley_initiative_for_transparency_in_the_social_sciences_guide_2022} This is copied without changes (except conversion from Markdown to Latex). We thank the authors for sharing this resource under the Creative Commons Attribution-NonCommercial 4.0 International License, which allows the material to be copied with attribution for non-commercial purposes. It has been copied into the appendices for easy access by researchers following this protocol, to adapt from when communicating with authors.
\end{shaded}

This chapter contains guidance for constructive and respectful communication between reproducers and original authors. Exchanges that contain charged or adversarial language can damage professional relationships and hamper scientific progress. Janz and Freese (\href{https://www.mzes.uni-mannheim.de/openscience/wp-content/uploads/2019/01/Janz-Freese_-Good-and-Bad-Replications-1.pdf}{2019}) articulate two important steps reproducers can take to ensure that their interactions with original authors are constructive. We summarize and build on this approach below and encourage you to follow this guidance. Remember the \textbf{golden rule of reproductions} (and replications): \textit{treat others and their work, as you would like others to treat you
and your work!}

\textbf{1. Carefully and transparently plan your study.}

\begin{enumerate}[label=(\alph*)]
    \item Clearly state that you are conducting a reproduction of their original work.
    \item Explain why you have chosen this study.
    \item If you are not able to reproduce the results explain how ``far'' your results must deviate from the original work before claiming that the study could not be reproduced. Engage deeply with the substantive literature to ensure that your interpretation of differences between the original and reproduction is thorough and acceptable to other authors in the field.
\end{enumerate}

\textbf{2. Use professional and sensitive language. Discuss potential discrepancies between your work and the original paper, just like you might do for your own work.}

\begin{enumerate}[label=(\alph*)]
    \item Avoid binary judgments and statements like ``failed to reproduce.'' Instead, state clearly which results reproduced and which did not (e.g., ``we successfully reproduced X, but failed to reproduce Y'') unless you uncover apparent scientific misconduct (e.g., see \href{https://doi.org/10.31222/osf.io/qy2se}{Broockman, Kalla and Aronow, 2015}).
    \item Talk about \textit{the study, not the author} to avoid making it personal. Make clear what the positive contribution of the original article is. Consider sending a copy of your reproduction report to the original authors.
    \item Discuss what your reproduction contributes to the literature, and refrain from claiming to give the final answer to the question.
    \item For papers published five or more years ago, be mindful that norms for reproducibility have evolved since then.
    \item Remember, \textit{the goal is not to criticize previous work or hunt for errors, but to move the literature forward!}
\end{enumerate}

To help you put these recommendations into practice, we developed template language for common scenarios that reproducers and authors may encounter in their interactions. While we hope you find these useful, note that they are \textit{only recommendations}, and you are welcome to modify them based on your project's context and needs. Feel free to \href{mailto:acre@berkeley.edu}{contact us} if you need more guidance or would like to provide feedback on these materials.

\subsection{For reproducers contacting the authors of the original study}

Consider the following \textit{before} you contact the original authors:

\begin{enumerate}
    \item Search the SSRP for reproduction materials or records of past interactions with the authors of a given paper. For each reproduction, consult the “Author inquiries” column to see whether authors have indicated whether they are available for further inquiries.
    \item Carefully read all footnotes, appendices, tables, captions, etc., to learn if, how, and where reproduction materials are provided. Follow this \href{https://social-science-data-editors.github.io/guidance/Verification_guidance.html}{Data and Code Guidance} to determine whether you have everything before you start. A few things to consider:
    \begin{itemize}
        \item A \textit{Readme} file, if available, would be a good place to start.
        \item Check whether there are any restrictions on accessing the data or code, and whether there are instructions on how to access these files for the purpose of reproduction.
    \end{itemize}
    \item If a reproduction package is not readily available in the location where the article is published (e.g., the journal website), check the authors' websites, Dataverse profiles, or other relevant archives and/or data repositories like the \href{https://www.icpsr.umich.edu/icpsrweb/}{ICPSR Publications Related Archive}.
    \item If steps 1 and 2 don't yield anything, contact the corresponding author (copying the co-authors, if any), consolidating your requests into as few emails as possible. In your email, make sure to include the following details:
    \begin{itemize}
        \item Basic information about the paper to reproduce (include title, version, date, and a DOI (or just a URL));
        \item Context for the reproduction (as part of a class exercise, thesis, personal project, etc.) and a note that the outcome will be recorded on the \href{https://www.socialsciencereproduction.org/}{Social Science Reproduction Platform}(SSRP);
        \item Items from the reproduction package that are missing, as well as locations where you had (unsuccessfully) searched for them;
        \item Your use plan: Will the materials be used exclusively for this project? Ask for permission to share the data publicly.
        \item Right to consultation and results: Will you share the outcome of the reproduction with the original authors?
        \item A deadline to respond (we suggest at least two weeks).
    \end{itemize}
    \item Follow up if you don’t get a response within two weeks (or whatever deadline you set), and include any details or clarifications that were left out in your first email.
    \item Record the outcome of your interaction with the original author at the \textit{Select a Paper} stage on the SSRP. You can qualify the outcome as one of the following:
    \begin{itemize}
        \item A \textit{complete} reproduction package was provided;
        \item An \textit{incomplete} reproduction package was provided. You can also
    select one of the following reasons:
        \begin{itemize}
            \item Data is of sensitive, confidential, or proprietary character and cannot be shared;
            \item Data is of sensitive, confidential, or proprietary character, but access instructions were provided.
        \end{itemize}
    \item The author \textit{declined} to share the reproduction package; or
    \item The author \textit{did not respond} (including after a reminder was sent) within four weeks after the initial request.
    \end{itemize}
\end{enumerate}

\subsubsection{Contacting the original author(s) when there is no reproduction package}

\textbf{Template email:}

\begin{quote}
\textbf{Subject:} Reproduction package for
\texttt{{[}"Title\ of\ the\ paper"{]}}
\end{quote}

\begin{quote}
Dear
\texttt{{[}Title\ (e.g.,\ "Dr.")\ Last\ name\ of\ Corresponding\ Author{]}},

I am contacting you to request a reproduction package for your paper
titled \texttt{{[}Title{]}} which was published in
\texttt{{[}Journal{]}} in \texttt{{[}year{]}} (vol
\texttt{{[}volume{]}}, no. \texttt{{[}no.{]}}), \texttt{{[}link{]}}. A
reproduction package contains (raw and/or analytic) data, code, and
other documentation that makes it possible to reproduce the paper. Would
you be able to share any of these items?

I am a \texttt{{[}graduate\ student/postdoc/other\ position{]}} at
\texttt{{[}Institution{]}}, and I would like to reproduce the results,
tables, and other figures using the reproduction materials mentioned
above. I have chosen this paper because
\texttt{{[}add\ context\ for\ why\ you\ want\ to\ reproduce\ this\ particular\ paper\ using\ neutral\ language\ (e.g.,\ "This\ is\ a\ seminal\ paper\ in\ my\ field"),\ avoiding\ any\ statements\ that\ would\ put\ the\ respondent\ on\ the\ defensive{]}}.
Unfortunately, I was not able to locate any of these materials on the
journal website, Dataverse
\texttt{{[}or\ other\ data\ and\ code\ repositories{]}}, or your
website.

I will record the result of my reproduction attempt on the
\href{https://www.socialsciencereproduction.org/}{Social Science
Reproduction Platform} (SSRP), an open-source platform for
systematically conducting and recording reproductions. With your
permission, I will also record the materials you share with me, which
would allow access for other reproducers and avoid repeated requests
directed to you. Please let me know if there are any legal or ethical
restrictions that apply to any of the reproduction materials so that I
can take that into consideration during this exercise.

In addition to your response above, would you be available to respond to
future (non-repetitive) inquiries from me or other SSRP users? Though
your cooperation with my and/or future requests would be extremely
helpful, please note that you are \textit{not required to respond}.

Since I am required to complete this project by \texttt{{[}date{]}}, I
would appreciate your response by \texttt{{[}deadline{]}}.

Let me know if you have any questions. Please also feel free to contact
my supervisor/instructor \texttt{{[}Name\ (email){]}} for further
details on this exercise. Thank you in advance for your help!

Best regards,\\
\texttt{{[}Reproducer{]}}
\end{quote}

\subsubsection{Follow-up if the author only provides the
appendix}

If the original author asks you to reproduce the results using the appendix, you are not obligated to undertake that effort. Should you choose to do so, you can send a follow-up email as follows:

\begin{quote}
Dear
\texttt{{[}Title\ (e.g.,\ "Dr.")\ Last\ name\ of\ Corresponding\ Author{]}},

Thank you for your response. The purpose of my reproduction is to assess
and improve computational reproducibility using the original data and
code. Your appendix was very helpful, and I have attempted to use it to
reproduce your
results.\texttt{{[}Describe\ initial\ efforts\ to\ reproduce\ the\ results.{]}}.

To help me advance with the reproduction, I hope you can share the
source code and provide guidance on the following:
\texttt{{[}List\ out\ any\ unclear\ steps,\ data\ sources,\ or\ otherwise\ missing\ components.\ Use\ bullet\ points\ if\ more\ than\ two{]}}.

Thank you in advance for your help! Once completed, I will make the
reproduction package publicly available to be used in future SSRP
reproductions. Please let me know if you have any questions.

Best regards,\\
\texttt{{[}Reproducer{]}}
\end{quote}

\subsubsection{Contacting the original author(s) to request specific
missing items of a reproduction
package}

\textbf{Template email:}

\begin{quote}
\textbf{Subject:} Reproduction materials for
\texttt{{[}"Title\ of\ the\ paper"{]}}
\end{quote}

\begin{quote}
Dear
\texttt{{[}Title\ (e.g.,\ "Dr.")\ Last\ name\ of\ Corresponding\ Author{]}},

I am contacting you regarding reproduction materials for your paper
titled \texttt{{[}Title{]}} was published in \texttt{{[}Journal{]}} in
\texttt{{[}year{]}} (vol \texttt{{[}volume{]}}, no. \texttt{{[}no.{]}}),
\texttt{{[}link{]}}.

I am a \texttt{{[}graduate\ student/postdoc/other\ position{]}} at
\texttt{{[}Institution{]}}, and I'm working on reproducing this paper as
part of a class assignment.
\texttt{{[}Add\ context\ for\ why\ you\ want\ to\ reproduce\ this\ particular\ paper\ using\ neutral\ language\ (e.g.,\ "This\ is\ a\ seminal\ paper\ in\ my\ field"),\ avoiding\ any\ statements\ that\ would\ put\ the\ respondent\ on\ the\ defensive{]}}.

To help me reproduce the paper in full, I hope that you can share the
following items:
{[}\texttt{list\ items\ missing\ from\ reproduction\ package,\ preferably\ bulleted\ if\ more\ than\ one\ (e.g.,\ raw/analytic\ data,\ code,\ protocols\ for\ conducting\ the\ experiment,\ etc.)}{]}.
I have already searched
\texttt{{[}locations\ where\ you\ searched\ for\ items,\ with\ links\ provided{]}},
but I could not locate the items. Unless you tell me otherwise, I will
make the reproduction package publicly available to be used in future
reproductions. Let me know if any legal or ethical restrictions apply to
any of the reproduction materials to consider during this exercise.

Note that I will record the outcome of my reproduction on the
\href{https://www.socialsciencereproduction.org/}{Social Science
Reproduction Platform}(SSRP), an open-source platform for systematically
conducting and recording reproductions. Let me know if you would like me
to share the outcome of my reproduction with you, and whether you are
interested in providing a response.

Since I am required to complete this project by \texttt{{[}date{]}}, I
would appreciate your response by \texttt{{[}deadline{]}}.

Let me know if you have any questions. Please also feel free to contact
my supervisor/instructor \texttt{{[}Name\ (email){]}} for further
details on this exercise. Thank you in advance for your help!

Best regards,\\
\texttt{{[}Reproducer{]}}
\end{quote}

\subsubsection{Asking for additional guidance when some materials have
been shared}

\textit{Note:} Even when a corresponding author has shared a reproduction
package, you may still face challenges in interpreting or executing the
materials. That shouldn't discourage you from asking the corresponding
author to provide clarifications or to share the missing materials. As
in the previous scenario described above, demonstrate that you've made
an honest effort to reproduce the work using the available resources and
try to consolidate your requests in as few emails as possible.

\textbf{Template email:}

\begin{quote}
\textbf{Subject:} Clarification for reproduction materials for
\texttt{{[}"Title\ of\ the\ paper"{]}}
\end{quote}

\begin{quote}
Dear
\texttt{{[}Title\ (e.g.,\ "Dr.")\ Last\ name\ of\ Corresponding\ Author{]}},

Thank you for sharing your materials. They have been immensely helpful.

Unfortunately, I ran into a few issues as I delved into the
reproduction, and I think your guidance would be helpful to resolve
them.
\texttt{{[}Describe\ the\ issues\ and\ how\ you\ have\ tried\ to\ resolve\ them.\ Describe\ whatever\ files\ or\ parts\ of\ the\ data\ or\ code\ are\ missing.\ Refer\ to\ examples\ 1\ and\ 2\ below\ for\ more\ details{]}}.

Thank you in advance for your help.

Best regards,\\
\texttt{{[}Reproducer{]}}
\end{quote}

\textbf{1: An example of a well-described issues:}

\begin{quote}
Specifically, I am attempting to reproduce Display Item X (e.g., table 1, figure 3). I found that the following components are required to reproduce Display Item X:

\texttt{{[}Tree of files needed to produce item{]}}.

I have marked with an asterisk the items that I could not find in the
reproduction materials: \textbf{data\_cleaning01.R} and
\textbf{admin\_01raw.csv}. After accessing these files, I will also
identify the name of the raw data set required to obtain
output1\_part1.txt. This is to let you know that I may need to contact
you again if I cannot find this file (labeled as UNKNOWN above) in the
reproduction materials.

I understand that this request will require some work for you; I will
publish the materials on the
\href{https://www.socialsciencereproduction.org/}{Social Science
Reproduction Platform}, which should help avoid repeated requests in the
future.
\end{quote}

\textbf{2. An example of a poorly-described issue:}

\begin{quote}
Your paper does not reproduce. I have tried for several hours now, and
can't get the DO files to run. Could you please share all the missing
reproduction materials? Data and code sharing are basic principles of
open science, so I am confident that you will do the right thing.
\end{quote}

\subsubsection{\texorpdfstring{Response when the original author has
declined to share due to \textit{undisclosed
reasons}}{Response when the original author has declined to share due to undisclosed reasons}}

\textit{Note:} You can also use this template if a corresponding author
has not submitted a response after two or more follow-up emails.

\textbf{Template email:}

\begin{quote}
\textbf{Subject:} Re: Reproduction materials for
\texttt{{[}"Title\ of\ the\ paper"{]}}
\end{quote}

\begin{quote}
Dear
\texttt{{[}Title\ (e.g.,\ "Dr.")\ Last\ name\ of\ Corresponding\ Author{]}},

Thank you for considering my request. I will try to reproduce the paper
using the available materials and will record the missing items
accordingly on the
\href{https://www.socialsciencereproduction.org/}{Social Science
Reproduction Platform} (SSRP). l will also post my assessment of the
reproducibility of the paper in its current form based on the
\href{https://bitss.github.io/ACRE/assessment.html\#levels-of-computational-reproducibility-for-a-specific-output}{SSRP
reproducibility scale}.

Let me know if you have any questions.

Best regards,\\
\texttt{{[}Reproducer{]}}
\end{quote}

\subsubsection{Response when the original author has declined to share
due to legal or ethical restrictions of the
data}

\textbf{Template email:}

\begin{quote}
\textbf{Subject:} Re: Reproduction materials for
\texttt{{[}"Title\ of\ the\ paper"{]}}
\end{quote}

\begin{quote}
Dear
\texttt{{[}Title\ (e.g.,\ "Dr.")\ Last\ name\ of\ Corresponding\ Author{]}},

Thank you for your response and for clarifying the terms of use for the
reproduction materials.

Though I understand you are unable to share the raw data, there may be
alternative steps you can take that would help me improve the
reproducibility of your paper. These include:

\begin{enumerate}
\def\labelenumi{\arabic{enumi}.}
\tightlist
\item
  Sharing the analytic version of the data (the version of the dataset
  that was used for analysis in the final version of your paper);\\
\item
  Providing a public description of the steps other researchers can
  follow to request access to the raw data or materials, including an
  estimate of the costs and the duration of the process. You can find
  examples of data availability statements for proprietary or
  restricted-access data
  \href{https://social-science-data-editors.github.io/guidance/Requested_information_dcas.html}{here};
  and\\
\item
  Providing access to all data and materials for which the constraints
  do not apply.
\end{enumerate}

Based on my assessment, your paper would currently rank at
\texttt{{[}level\ X{]}} on the
\href{https://bitss.github.io/ACRE/assessment.html\#levels-of-computational-reproducibility-for-a-specific-output}{SSRP
reproducibility scale}. However, \textit{this score can be easily
improved}. Being able to provide analytic data would elevate the
reproducibility of your paper to \texttt{{[}level\ Y{]}}. Providing
public instructions on how other parties can access the data would
further elevate its reproducibility to \texttt{{[}level\ Z{]}}.

I would be happy to help if you are interested in taking any of the
steps I outlined above.

Thank you for your help!

Best regards,\\
\texttt{{[}Reproducer{]}}
\end{quote}

\subsubsection{Contacting the original author to share the results of your reproduction exercise}

\textit{Note}: Reporting the results of reproductions can be the most
contentious part of the process, particularly when the reproducer is
unable to fully reproduce the paper or finds significant deviations from
the original work. However, if the reproduction can correctly identify
the sources of such deviations, it may be viewed as an improved version
of the original work.

Regardless of the outcome of the reproduction exercise, the guidance
from the introduction of this chapter still stands: \textit{reproduce the
work of others as you would like for others to reproduce yours}, and
make sure that is reflected in how you discuss any discrepancies between
your and the original work.

\textbf{Template email:}

\begin{quote}
\textbf{Subject:} Reproducibility Assessment of
\texttt{{[}"Title\ of\ the\ paper"{]}}
\end{quote}

\begin{quote}
Dear
\texttt{{[}Title\ (e.g.,\ "Dr.")\ Last\ name\ of\ Corresponding\ Author{]}},

Thank you for your support throughout my project as I worked to verify
and advance the reproducibility of \texttt{{[}Paper{]}}. I'm writing now
to share the results of my project and to invite your feedback.

The results of each step of my reproduction include i) Assessment, ii)
Improvements, iii) Robustness Checks, (and iv) Extensions, if
applicable).\\
`{[}Include the following items in the body of your email:

\begin{itemize}
\tightlist
\item
  Briefly describe which parts of the paper you tried to reproduce
  (e.g., a specific estimate, a table, etc.).\\
\item
  Within the scope of your reproduction, describe exactly which items
  you were able to reproduce.\\
\item
  Discuss the differences you observed between the results of your
  reproduction and the original work, and demonstrate that you did your
  due diligence in trying to reproduce each item. Remember that it is
  more constructive to discuss discrepancies, differences or deviations,
  rather than errors, mistakes, or failures, and \textit{always talk about
  the work---not the author!}\\
\item
  Use sensitive language when presenting discrepancies, e.g.,
  ``Unfortunately, I found X, which differs from the Y result in the
  original paper\ldots{}''. Be mindful of any potential limitations of
  your work, and explain how you have tried to address them---that way,
  you will proactively address potential criticism!\\
\item
  Describe how you tried to improve the reproducibility of the paper. If
  some of the improvements are based on discretionary judgment (e.g.,
  file organization or code commenting), try to explain why you think
  they are an improvement over the original work. If you didn't make
  improvements, point out some concrete steps that the author(s) can
  take to improve the reproducibility of the section you reproduced.{]}`
\end{itemize}

I look forward to your questions, comments, and suggestions for my work.
As discussed previously, I will record the outcomes of my reproduction,
along with the improvements, on the
\href{https://www.socialsciencereproduction.org/}{Social Science
Reproduction Platform}.

Best regards,\\
\texttt{{[}Reproducer{]}}
\end{quote}

\subsubsection{Responding to hostile responses from original
authors}

\textit{Note:} Planning your study carefully and transparently and using
professional and sensitive language are the best ways to ensure that the
interaction will be beneficial to both you and the original author.
However, unpleasant interactions may happen despite your best efforts
and can range anywhere from dismissive comments to bullying,
discrimination, and harassment. Find guidance at the end of this chapter
on how to deal with instances of bullying, harassment, or
discrimination.

\textbf{Dismissive comments:} In cases of dismissive comments, the best course of action may be to simply thank the author for their response and continue with the reproduction.

\textbf{Template email:}

\begin{quote}
\textbf{Subject:} Re: Reproduction materials for
\texttt{{[}"Title\ of\ the\ paper"{]}}
\end{quote}

\begin{quote}
Dear
\texttt{{[}Title\ (e.g.,\ "Dr.")\ Last\ name\ of\ Corresponding\ Author{]}},

Thank you for your response. I will work to reproduce your paper using
the available materials and will record my results accordingly on the
\href{https://www.socialsciencereproduction.org/}{Social Science
Reproduction Platform}. l will also post my assessment of the
reproducibility of the paper in its current form based on the
\href{https://bitss.github.io/ACRE/assessment.html\#levels-of-computational-reproducibility-for-a-specific-output}{SSRP
reproducibility scale}.

Let me know if you have any questions.

Best regards,\\
\texttt{{[}Reproducer{]}}
\end{quote}

\subsection{For original authors responding to requests from reproducers}

This section contains guidance for authors whose work is involved in
reproductions on the Social Science Reproduction Platform. We present
language that may help various scenarios in which authors find
themselves when interacting with reproducers. Though every interaction
between authors and reproducers takes place is distinct and may carry
its unique challenges, the guiding principle of
\href{https://bitss.github.io/ACRE/guidance-for-a-constructive-exchange-between-reproducers-and-original-authors.html}{this
chapter} always applies: ``Treat others and their work as you would like
others to treat you and your work!'' We hope that these resources will
facilitate more efficient and constructive exchanges between the parties
involved. \href{emailto:acre@berkeley.edu}{Let us know} if you need
guidance in other scenarios!

\subsubsection{Responding to a repeated request that has been addressed in an earlier interaction}

\begin{quote}
Dear {[}Reproducer{]},

Thank you for your interest in my work. I have been contacted about this
issue by another SSRP reproducer before and provided a response, which I
suspect may be already recorded on the SSRP. I'm copying my original
response below for your reference. You may find further guidance in the
readme file in the reproduction package.

If there are no prior records of these issues on the SSRP, please record
the enclosed response {[}and materials{]}. This will also help avoid the
duplication of effort on the part of others who may be interested in
reproducing this work.

Good luck with the remainder of your project and thank you in advance
for your cooperation!

Best regards, {[}Author{]}
\end{quote}

\subsubsection{Acknowledging that the author no longer has access to certain part(s) of the reproduction package}

\begin{quote}
Dear {[}Reproducer{]},

Thank you for reviewing my work closely. I wish I could be of more help,
but unfortunately I no longer have access to the requested materials due
to
\texttt{{[}briefly\ describe\ the\ circumstances\ that\ prevent\ you\ from\ providing\ the\ materials{]}}.

While I recognize that the current standards in the discipline have
moved towards computational reproducibility, note that this paper was
written when different standards applied. Please feel free to evaluate
the paper as is and propose any improvements wherever possible.

I look forward to working with you to address this and improve the
overall reproducibility of the paper.

Best regards, {[}Author{]}
\end{quote}

\subsubsection{Acknowledging that some material is still embargoed for future research}

\begin{quote}
Dear {[}Reproducer{]},

Thank you for your interest in my work. The data/materials/program that
you reference are currently not publicly available because they are
embargoed until \texttt{{[}embargo\ period{]}}.

\texttt{{[}Depending\ on\ the\ restrictions\ that\ apply\ to\ the\ reproduction\ package,\ consider\ alternatives\ to\ sharing\ the\ reproduction\ materials\ in\ full.\ These\ include:\ 1.Sharing\ the\ analytic\ version\ of\ the\ data\ (the\ version\ of\ the\ dataset\ that\ was\ used\ for\ analysis\ in\ the\ final\ version\ of\ your\ paper);\ \ 2.\ Providing\ a\ public\ description\ of\ the\ steps\ that\ other\ researchers\ can\ follow\ to\ request\ access\ to\ the\ raw\ data\ or\ materials,\ including\ an\ estimate\ of\ the\ costs\ and\ the\ duration\ of\ the\ process.\ Find\ examples\ of\ data\ availability\ statements\ for\ proprietary\ or\ restricted-access\ data\ {[}here{]}(https://social-science-data-editors.github.io/guidance/Requested\_information\_dcas.html);\ and\ 3.\ Providing\ access\ to\ all\ data\ and\ materials\ for\ which\ the\ constraints\ do\ not\ apply.{]}}

I hope you find this helpful. Please feel free to contact me if you have
any further questions.

Best regards, {[}Author{]}
\end{quote}

\subsubsection{Responding to incomplete/unclear
requests}

\begin{quote}
Dear {[}Reproducer{]},

Thank you for your interest in my work. I would be happy to assist you
and other reproducers to assess and improve the reproducibility of this
paper.

To help me give more concrete guidance, I'd appreciate if you could
provide a more specific description \textgreater of the items that you
need from me. You can find helpful information and resources in Chapter
6 of the Guide, \textgreater specifically
\href{https://bitss.github.io/ACRE/guidance-for-a-constructive-exchange-between-reproducers-and-original-authors.html\#asking-for-additional-guidance-when-some-materials-have-been-shared}{here}
\texttt{{[}based\ on\ the\ context,\ you\ may\ need\ to\ point\ \textgreater{}the\ reproducer\ to\ a\ \textgreater{}different\ scenario\ and/or\ provide\ further\ information{]}}.

Feel free to contact me if you have any further questions. Thank you for
your cooperation.

Best regards, {[}Author{]}
\end{quote}

\subsection{Harassment and/or
discrimination}
The American Economic Association (AEA) and other academic societies
have strict policies against harassment and discrimination. Here are
some of the behaviors that the
\href{https://www.aeaweb.org/about-aea/aea-policy-harassment-discrimination}{AEA
Policy on Harassment and Discrimination} has listed as unacceptable and
could emerge in a hostile exchange regarding a reproduction:

\begin{itemize}
\tightlist
\item
  Intentionally intimidating, threatening, harassing, or abusive actions
  or remarks (both spoken and in other media)
\item
  Prejudicial actions or comments that undermine the principles of equal
  opportunity, fair treatment, or free academic exchange
\item
  Deliberate intimidation, stalking, or following
\item
  Real or implied threat of physical harm.
\end{itemize}

Here are a some steps you can take if you believe you have experienced
bullying, discrimination or harassment:

\begin{itemize}
\tightlist
\item
  \textbf{File a complaint with the
  \href{https://www.aeaweb.org/about-aea/aea-ombudsperson}{AEA
  Ombudsperson}.} Any AEA member can file a complaint. You can also join
  the AEA solely to file a report. The person about whom you are making
  the complaint need not be an AEA member. A non-AEA member can also
  file a report if the act of harassment or discrimination was committed
  by an AEA member or in the context of an AEA-sponsored activity. Learn
  more about the process
  \href{https://www.aeaweb.org/about-aea/aea-ombudsperson/faq}{here}.
\item
  \textbf{File a report with your institution's office for the
  prevention of harassment \& discrimination.} US-based institutions
  have internal mechanisms that allow students and faculty to seek
  support in cases of discrimination and harassment based on race,
  color, national origin, gender, age, or sexual orientation/identity,
  including allegations of sexual harassment and sexual violence. Formal
  titles of this office vary across institutions, but common names
  include ``Office for the Prevention of Harassment and Discrimination''
  (in institutions that are part of the University of California
  system), ``Office of Equity and Title IX,'' etc.
\item
  \textbf{Contact your institution's Ombudsperson/Ombuds Office.} If you
  believe that you have experienced academic bullying or other forms of
  disrespectful behavior that fall outside the scope of harassment
  and/or discrimination as described above, you should know that
  university ombuds officers offer a confidential, impartial resource to
  discuss your concerns and learn about potential next steps available
  in your case.
\item
  \textbf{Access mental health services at your institution.} While no
  amount of bullying, discrimination, or harassment is acceptable or the
  fault of the victim, these unfortunately still occur and can take a
  toll on victims' mental health. Many universities offer short-term
  Counseling \& Psychological Services (CAPS) for academic, career, and
  personal issues.
\item
  \textbf{Ask for support from your academic supervisor.} If you are
  unsure on how to proceed, consult your academic supervisor on whether
  continuing the reproduction is appropriate.
\end{itemize}