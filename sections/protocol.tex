\textbf{TO DO:} Confirm appropriate author order/inclusion/etc.

% Rename the abstract
% \renewcommand{\abstractname}{}
\begin{shaded}
    \begin{abstract}
        This protocol outlines how we plan to reuse available artifacts to reproduce results from published simulation studies. This forms part of the project STARS: "Sharing Tools and Artefacts for Reusable Simulations in healthcare". It will be utilised to conduct reproducibility tests on published and newly developed simulation models in Python and R.
    \end{abstract}
\end{shaded}

\section{Selection of simulation models}

\textbf{TO DO:} Describe selection of models for WP1, because we want to be transparent about how choosing the models to reproduce. For WP3, just mention current plans roughly.

\textbf{Rough notes:} Six published discrete-event simulation studies. These will be selected from the N studies with available code identified in the systematic scoping review conducted by Monks and Harper (REF). The studies will require an open license (will inequire to authors) and want balance of python and R. Two new models developed within/with support from the team, and with a team member acting as a holdout to conduct the reproducibility test.

\section{Reproduction of simulation models}

The protocol for the reproduction of simulation models is primarily adapted from:
\begin{itemize}
    \item Krafczyk et al. 2021 - Learning from reproducing computational results: introducing three principles and the Reproduction Package\cite{krafczyk_learning_2021}
    \item Wood et al. 2017 - Replication protocol for push button replication\cite{wood_replication_2018}
\end{itemize}

\textbf{TO DO:} Write this section