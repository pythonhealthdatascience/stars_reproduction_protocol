\section{Reproduction test reports}

\hlblue{\textbf{Grant:} Write RCR reports. \textbf{To discuss though!:} Got lots and lots of examples of write ups for this sort of thing, and identified opportunities for publishing individually (ReScience)}

\textbf{TO DO:} Write this section. Need to reflect on whether it can consistute an RCR review or not. And what we want to do with it, or if just archived, or if plan to write up in some way or another.

Tom: Finally, we will map data from our best practice review, reporting guidelines adherence, and our tracked test data to success in badge allocation and the proportion of experiments reproduced.

Produce a plot reporting the percentage of experiments reproduced versus time. - but need to consider this based on how define reproduction success

Final report:
\begin{itemize}
    \item Figure, and describe reproduction team, similar to Krafcyzk(?):\autocite{krafczyk_learning_2021} Krafczyk present time taken to produce each table/figure. Consider approach to this given: * Won't do a yes/no for reproduction, but more nuanced info on it for each item * Not just tables and figures - "key results" - which could be a table/figure from appendices, but could also be specific figures from the text. * A study with more figures and tables will take longer (which is not presented in that figure as its percentage of them done)
    \item Look at examples I've gathered
\end{itemize}

\section{Versioning and Archiving}

This reproduction work will be made openly available throughout the project using GitHub. It will also be archived in Zenodo upon completion.

\textbf{TO DO:} Is there any exception here with the NHS Somerset work, in terms of how will work?

This protocol itself will also be archived as a pre-registration. Haroz 2022 identifies the Open Science Framework (OSF, \url{https://osf.io/}) and Zenodo (\url{https://zenodo.org/}) as suitable platforms for pre-registration.\autocite{haroz_comparison_2022} In this case, Zenodo will be used as this is where other materials already exist for the STARS project, and so it can be stored alongside them in a Zenodo "community".

\textbf{TO DO:} Maybe remove this section, and just mention abut GitHub and Zenodo above in record keeping and reports, and don't need to mention Haroz?

Rough notes from Tom:

Will use GitHub + Zenodo (automated via GitHub actions)
Will containerise with Docker or similar (again, automated via GitHub actions)
Will share via GitHub pages
MIT and CC-BY-4.0 licenses
Online book will describe how to reuse, adapt and reshare our outputs

Create standardised structure for report, and include stuff from the original article within it.

Report should have things like Baykova \autocite{baykova_ensuring_2024} like:
\begin{itemize}
    \item Title of manuscript
    \item Corresponding author
    \item Summary of paper
    \item Available manuscript materials DOI/URL... pre-registratin... manuscript... data and analysis materials
    \item Scope of reproducibiltiy report
    \item Date it was compiled, what it was based on (e.g. draft, pre-print), materials uploaded to (link) as of (date and time of upload)
    \item "I certify that to the best of my knowledge this report is true and accurate"
    \item Who did the report, email, address
\end{itemize}

\section{Testing the protocol}

Test with Tom's paper.

Also maybe IPACS?

Think about what test and why. Is it run through of whole thing, or just particular bits.
