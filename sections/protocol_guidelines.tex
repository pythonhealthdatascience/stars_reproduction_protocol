\section{Evaluation against guidelines}

\subsection{Badges} \label{sec:badges}
\timeno

Various organisations and journals have developed badges which can be displayed alongside a published research article to indicate how open and potentially reproducible it is, as detailed in Appendix \ref{appendix:badges}.

For this study, the researcher should:
\begin{enumerate}
    \item Evaluate the artefacts \textbf{made available by the author} (code, data, documentation) against the criteria of different badges (as in Tables 1 to 3 in Appendix \ref{appendix:badges}). \textit{To be clear, this is not evaluating the repository created by the researcher whilst reproducing the study, but just the original author artefacts.}
    \item Identify which of the badges it meets the criteria for.
\end{enumerate}

Within our Quarto template (\url{https://github.com/pythonhealthdatascience/stars_reproduction_template}), a skeleton page is provided for completion of this task.\hl{make that page}

\subsection{Sharing research artefacts} \label{sec:artefacts}
\timeno

Note if changes are made by author after having being contacted by us (e.g. they added an open license after we asked them to) \hl{although what about WP3? when it's all us having told them?}

STARS Essential + Enhanced

Any other major or relevant to DES.

\hl{there is considerable overlap between this and badges actually... not sure if we want this.}

\hl{could do this first then badges as that order might make more sense?}

\subsection{Reporting guidelines} \label{sec:reporting}
\timeno

The journal article will be evaluated against two reporting guidelines for discrete-event simulation studies:
\begin{itemize}
    \item STRESS-DES: Strengthening The Reporting of Empirical Simulation Studies (Discrete-Event Simulation)\autocite{monks_strengthening_2019}
    \item The generic reporting checklist for healthcare-related discrete event simulation studies derived from the the International Society for Pharmacoeconomics and Outcomes Research Society for Medical Decision Making (ISPOR-SDM) Modeling Good Research Practices Task Force reports.\autocite{zhang_reporting_2020}
\end{itemize}

We will score items as fully, partially, or not meeting the requirements - or as unclear (if required information to assess guideline criteria is not included). Within our Quarto template (\url{https://github.com/pythonhealthdatascience/stars_reproduction_template}), a skeleton page is provided for completion of this task.\hl{make that page}

In the second part of our study, where we assess the computational reproducibility of two simulation models developed alongside the STARS team, we may potentially include a simulation that is not discrete-event. In that case, we will identify alternative appropriate reporting guidelines to evaluate against, for that model type. For example, the Overview, Design concepts and Details (ODD) provides guidelines for describing individual or agent-based models,\autocite{grimm_odd_2020} or the Consolidated Health Economic Evaluation Reporting Standards (CHEERS) checklist.\autocite{husereau_consolidated_2013, husereau_consolidated_2013-1}