\section{Evaluation against guidelines}

This section is completed after the reproducibility assessment (so as to not interfere with timings).

\textbf{Remember!} Record progress in your logbook - but no need to time each task.

\textbf{Important!} This evaluation is based on the original journal article or repository from the author (as in original\_study/), and not on the repository you made whilst reproducing this study (reproduction/). If the original study had multiple repositories to choose from (e.g. code repository and archived code, both prior to publication date), remember to refer to both of them if there are any differences between them.

\textbf{Handy tip!} For each those these evaluations, if you are uncertain on any criteria, make a note of these within the logbook and discuss with the wider STARS team to get a second opinion. Likewise, for the reporting guidelines, when an item is "not met" we suggest what we think the answer might be if possible, and it is good to double-check these with the team also.

\vspace{0.5cm}
\subsection{Best practice for sharing of research artefacts}

The artefacts (repository) associated with the original study will be evaluated against two sets of criteria/ recommendations on the sharing of research artefacts for simulation models:
\begin{itemize}
    \item Criteria from the \textbf{best practice audit} conducted by Monks and Harper 2023\autocite{monks_computer_2023} as part of their review. The audit is described in detail in the repository associated with their review.\autocite{monks_computer_2024} The items used in this audit were based on guidance from the Turing Way,\autocite{the_turing_way_community_turing_2022} Taylor et al. 2017,\autocite{taylor_open_2017} and the Open Modelling Foundation (OMF) minimal and ideal reusability standards,\autocite{the_open_modeling_foundation_omf_reusability_2024} focussing on items that were relevant to the modelling and simulation community.
    \item The \textbf{STARS framework} (developed following the above review) recommends essential and optional components when sharing healthcare simulation studies.
\end{itemize}

You should use the \textbf{provided template} (evaluation/artefacts.qmd) to assess whether the artefacts from the original study met the criteria/recommendations from each of these sources. Each criteria are evaluated as being "fully", "partially" or "not met".

For the best practice audit, all studies will have previously been evaluated by Monks and Harper 2023\autocite{monks_computer_2023} After you have done your assessment, you should add the results from their review of that study to the logbook and compare against your assessment, as this can provide a simple sense-check/second opinion.

\vspace{0.5cm}
\subsection{Badges} \label{sec:badges}

Several organisations and journals have developed badges which can be displayed alongside a research article to indicate how open and potentially reproducible it is, as detailed in Appendix \ref{appendix:badges}. We will the original study artefacts (repository) against badges that relate to code (and not those specific to data), due to the nature of DES models (where "data" is often just parameters as part of the model script, with perhaps a few additional parameters in a separate data file within the repository). These badges are:
\begin{itemize}
    \item "Open objects" badges: NISO "Open Research Objects (ORO)", ACM "Artifacts Available", COS "Open Materials" and "Open Code", and IEEE "Datasets Available"
    \item "Object review" badges: ACM "Artifacts Evaluated" (rated as "Functional or "Reusable") and IEEE "Code Reviewed"
    \item "Reproduced" badges: NISO "Results Reproduced (ROR-R)", ACM "Results Reproduced", IEEE "Code Reproducible" and Psychological Science "Computational Reproducibility"
\end{itemize}

You should use the \textbf{provided template} (evaluation/badges.qmd) to assess whether the artefacts from the original study meet the criteria for each of these badges. A binary decision is made for each criteria (as being either met or not met).

\subsection{Reporting guidelines} \label{sec:reporting}

The journal article will be evaluated against two reporting guidelines for discrete-event simulation studies:
\begin{itemize}
    \item STRESS-DES: Strengthening The Reporting of Empirical Simulation Studies (Discrete-Event Simulation)\autocite{monks_strengthening_2019}
    \item The generic reporting checklist for healthcare-related discrete event simulation studies derived from the the International Society for Pharmacoeconomics and Outcomes Research Society for Medical Decision Making (ISPOR-SDM) Modeling Good Research Practices Task Force reports.\autocite{zhang_reporting_2020}
\end{itemize}

You should use the \textbf{provided template} (evaluation/reporting.qmd) to assess whether the criteria from these guidelines are met by the journal article (including the supplementary material, although not including the code unless the article specifically refers to it for providing particular information). Each criteria are evaluated as being "fully", "partially" or "not met", with detailed evidence provided to support these claims (such as quotations from the article).

We provide evidence for each criteria (such as with quotes from the original study). If a criteria is not met by the original study, you are welcome to make a suggestion in the evidence column of what you think the likely answer for that criteria might be.