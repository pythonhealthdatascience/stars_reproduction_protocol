\section{Introduction}

In this protocol, we are focused on the \textbf{computational reproducibility} of simulation models. This is defined as the ability to get consistent results with a prior study when using  the same data and methods as that study. We are focusing on models developed using \textbf{Python and R}, as these are popular free and open-source software (FOSS) for the development of models like discrete-event simulation (DES).\autocite{monks_computer_2023}

The protocol will be used at two points during the STARS project:

\begin{enumerate}
    \item \textbf{Six external models from Monks and Harper 2023:} We will assess the computational reproducibility of six DES models. These are selected from the models identified by Monks and Harper 2023.\autocite{monks_computer_2023} The selection criteria are that: (i) the model code is publicly available, (ii) the model is created using Python or R, and (iii) the code has an open license (either already published, or added upon request from the STARS team).
    \item \textbf{Two models developed alongside collaborators:} We will be working alongside two teams to support them in applying the STARS framework either retrospectively or prospectively to their simulation models. Once this has been completed, a member of the team will assess the computational reproducibility of these models.\hl{Would require adaptation (i.e. won't have a publication to use to define scope, or might be private so can't publicly have Quarto book, and we'll have been involved so it won't be independent, and so on - so is it a separate protocol? or plan that now too?)}
\end{enumerate}

Throughout the study, results will be openly available and shared via a \textbf{Quarto website}. This will compile information on the reproduction of the article. This includes the notebooks (.ipynb or .Rmd) producing the items in the scope, as well as a chronological log of work using Quarto blog posts, and then later, the reproduction report and detailed study results.

\vspace{1cm}
\section{Logbook}

Throughout \textbf{all of the stages} below, you should keep a \textbf{logbook}. This is a detailed record of work recorded using \textbf{Quarto blog posts}, using the template provided. As suggested by Ayllón et al. 2021\autocite{ayllon_keeping_2021} in their guidelines for keeping modelling notebooks, these posts will be \textbf{daily}, dated, chronological entries. \textbf{Tags} will be used to help indicate the activity on each day, and enable posts to be filtered by activity (although tags are free to be chosen by the researcher not related to a particular framework). Keeping a detailed log will support later understanding of what was done, and support preparing of final documents like the reproduction report. Each entry in the logbook should contain the:
\begin{itemize}
    \item \textbf{Researcher name} and \textbf{date}
    \item \textbf{Time} spent on tasks (if applicable)
    \item \textbf{Comprehensive} record of work. This should include record of working through each stage in the protocol, detailing \textbf{successes} and any \textbf{issues} faced, any \textbf{solutions} found to problems, and any \textbf{changes made to the model code} (noting where and how the code was changed). It may be relevant to include links to particular versions of a file or repository such as via the Git commit history.
    \item Clear statement if and when each item in the scope is considered to have been \textbf{successfully reproduced}.
\end{itemize}

\vspace{1cm}
\section{Timing}

You'll see that there are three core stages to this protocol:
\begin{itemize}
    \item Assessment of computational reproducibility
    \item Evaluation against guidelines
    \item Research compendium and report
\end{itemize}

For the first stage - \textbf{assessment of computational reproducibility} - you should \textbf{time how long each task takes}. These times should be recorded within the logbook alongside each activity (e.g. 12:10 to 12:45). The times should be monitored with a \textbf{maximum of 40 hours} allowed for attempting to reproduce the study, as in Krafczyk et al. 2021.\autocite{krafczyk_learning_2021} This cut-off is implemented as we anticipate there would be little more to learn from spending longer than that time on reproducing a single study. The timings are just implemented for the first stage, as our focus in timing this work is to understand (and ultimately also limit) how long we spend reproducing each study.

The only exceptions to this timing are:
\begin{itemize}
    \item \textbf{Computation time}. This is at the researcher's discretion. For example, you should include short run times where you are still continuously working on the study. You should exclude longer run times where you are no longer working towards the assessment of computational reproducibility - for example, if you set the simulation to run for five minutes whilst going to make a cup of tea.
    \item \textbf{Consensus on the scope}. You should record the time spent by the primary researcher in reading the article and identifying what they think is in scope. However, you should not include the time spent discussing the scope with other researchers on the team, nor any of the time that they spent reading the article or thinking about scope.
\end{itemize}