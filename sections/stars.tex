\section{Appendix: STARS framework} \label{appendix:stars}

The essential and optional components of the STARS framework are presented in Tables \ref{table:stars-essential} and \ref{table:stars-optional}, as from Monks et al. (2024).\autocite{monks_towards_2024}

\hl{add citations and abbreviations}

\begin{table}[H]
\centering
\caption{Essential components of STARS framework}
% Space between caption and table
\vspace{0.2cm}
% Identifier for the table
\label{table:stars-essential}
% Minimum column width
\tymin=3cm
% Add space between table rows
{\renewcommand{\arraystretch}{1.2}
    \begin{tabulary}{\linewidth}{@{}LL@{}}
        \toprule
        Component & Description
        \\\midrule
        Open license & Free and open-source software (FOSS) license (e.g. MIT, GNU Public License (GPL))
        \\\addlinespace
        Dependency management & Specify software libraries, version numbers and sources (e.g. dependency management tools like pip virtual environment, conda environment, poetry)
        \\\addlinespace
        FOSS model & Coded in FOSS language (e.g. R, Julia, Python)
        \\\addlinespace
        Minimum documentation & Minimal instructions (e.g. in README) that overview (a) what model does, (b) how to install and run model to obtain results, and (c) how to vary parameters to run new experiments
        \\\addlinespace
        ORCID & ORCID for each study author
        \\\addlinespace
        Citation information & Instructions on how to cite the research artefact (e.g. CITATION.cff file)
        \\\addlinespace
        Remote code repository & Code available in a remote code repository (e.g. GitHub, GitLab, BitBucket)
        \\\addlinespace
        Open science archive & Code stored in an open science archive with FORCE11 compliant citation and guaranteed persistance of digital artefacts (e.g. Figshare, Zenodo, the Open Science Framework (OSF), and the Computational Modeling in the Social and Ecological Sciences Network (CoMSES Net))
        \\\bottomrule
    \end{tabulary}
}
\end{table}

\begin{table}[H]
\centering
\caption{Optional components of STARS framework}
% Space between caption and table
\vspace{0.2cm}
% Identifier for the table
\label{table:stars-optional}
% Minimum column width
\tymin=3cm
% Add space between table rows
{\renewcommand{\arraystretch}{1.2}
    \begin{tabulary}{\linewidth}{@{}LL@{}}
        \toprule
        Component & Description
        \\\midrule
        Enhanced documentation & Open and high quality documentation on how the model is implemented and works  (e.g. via notebooks and markdown files, brought together using software like Quarto and Jupyter Book). Suggested content includes:\newline• Plain english summary of project and model\newline• Clarifying license\newline• Citation instructions\newline• Contribution instructions\newline• Model installation instructions\newline• Structured code walk through of model\newline• Documentation of modelling cycle using TRACE\newline• Annotated simulation reporting guidelines\newline• Clear description of model validation including its intended purpose
        \\\addlinespace
        Documentation hosting & Host documentation (e.g. with GitHub pages, GitLab pages, BitBucket Cloud, Quarto Pub)
        \\\addlinespace
        Online coding environment & Provide an online environment where users can run and change code (e.g. BinderHub, Google Colaboratory, Deepnote)
        \\\addlinespace
        Model interface & Provide web application interface to the model so it is accessible to less technical simulation users
        \\\addlinespace
        Web app hosting & Host web app online (e.g. Streamlit Community Cloud, ShinyApps hosting)
        \\\bottomrule
    \end{tabulary}
}
\end{table}