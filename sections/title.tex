\title{
    % Move higher on page
    \vspace{-1cm}
    % Include image then 1cm gap
    \includegraphics[width=9cm]{images/stars_logo_blue_text.png}\\[1cm]
    % Title
    \textbf{Protocol for assessing the computational reproducibility of simulation models on STARS}
}

% Authors
\author[1]{\orcidlink{0000-0002-6596-3479} Amy Heather}
\author[1]{\orcidlink{0000-0003-2631-4481} Thomas Monks}
\author[2]{\orcidlink{0000-0001-5274-5037} Alison Harper}
% Include line break (to prevent name breaking over two lines)
\author[2]{\\ \orcidlink{0000-0002-2204-8924} Navonil Mustafee}
\author[3]{\orcidlink{0000-0003-1263-2286} Andy Mayne}

% Affiliations (ensure update with numbers used above)
\affil[1]{\footnotesize University of Exeter Medical School, Exeter, UK}
\affil[2]{\footnotesize University of Exeter Business School, Exeter, UK}
\affil[3]{\footnotesize Taunton and Somerset NHS Foundation Trust, UK}

% No date
\date{}

\maketitle

\textbf{Publication date:} \hl{Add prior to submission}

% Add space before abstract
\vspace{0.5cm}

% Create abstract in a shaded box
\begin{shaded}
    \begin{abstract}
        This protocol outlines how we plan to reuse available artefacts to reproduce results from published simulation studies. This forms part of the project STARS: "Sharing Tools and Artefacts for Reproducible Simulations in healthcare". It will be utilised to assessed the computational reproducibility of published discrete-event (DES) simulation models in Python and R.
    \end{abstract}
\end{shaded}

% Add space after abstract
\vspace{0.5cm}

This protocol is archived as a pre-registration. Haroz 2022 identifies the Open Science Framework (OSF, \url{https://osf.io/}) and Zenodo (\url{https://zenodo.org/}) as suitable platforms for pre-registration.\autocite{haroz_comparison_2022} In this case, Zenodo will be used as this is where other materials already exist for the STARS project, and so it can be stored alongside them in a Zenodo "community".

\newpage
\tableofcontents